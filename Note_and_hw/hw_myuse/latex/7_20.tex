\documentclass[12pt, a4paper, oneside]{ctexart}
\usepackage{amsmath, amsthm, amssymb, graphicx,fancyhdr,geometry , subfigure}
\usepackage[bookmarks=true, colorlinks, citecolor=blue, linkcolor=black]{hyperref}
\geometry{a4paper,scale=0.75}
\pagestyle{fancy}
\fancyhf{} 
\fancyhead[L]{控制学院2024年暑期社会实践}
\fancyhead[C]{} % Center header - document title
\fancyhead[R]{京津线} % Right header - page number
\fancyfoot[C]{\thepage}

\begin{document}

\section{中国航天科工集团第二研究院}

\subsection{基本情况}

中国航天科工集团第二研究院(简称二院)创建于1957年11月16日,建院初期为国防部第五研究院二分院,现为中国空天防御技术总体研究院,是我国导弹工业的摇篮,我国航天事业和国防科技工业的中坚力量,航天强国建设和国防武器装备建设的主力军。

二院现有直属单位19家,在职职工24000余人,拥有6名两院院士、4名国家级有突出贡献专家和一大批国家级科技英才在内的知名专家。

二院坚持以习近平新时代中国特色社会主义思想为指导,秉承“国家利益高于一切”的核心价值观,履行“科技强军、航天报国”的神圣使命,发扬“求实、创新、协同、奉献”的企业精神,坚持“有矛必有盾”的发展哲学,深入贯彻落实习近平强军思想,确立建设世界一流航天安全防务研究院发展目标。

在履行强军使命的同时,二院全面贯彻新发展理念,融入服务新发展格局,按照“集约经营、集聚管理、集群发展”战略思路,推动产业高质量发展。聚焦新一代信息技术、高端装备制造、现代服务业三大主业,打造信息技术应用创新、雷达与电子信息、智慧产业等先导性产业,发挥航天技术优势构建民用产品体系。

\subsubsection{中国航天二院23所}
中国航天科工二院二十三所(以下简称二十三所)组建于1958年11月24日,是我国地空导弹制导雷达的摇篮,我国最高水平的制导雷达研究所。二十三所以雷达系统工程及电子信息技术为专长,构建了“一个核心、四大支柱、若干孵化”的产业发展格局,形成空天防御探测、信息支援保障和产业链延伸三大产品体系,覆盖制导雷达、预警雷达、态势感知雷达、测量雷达、气象雷达、空天基雷达、电子对抗装备等七大产品领域,先后承担我国多项重要武器装备研制生产任务。

二十三所现有从业人员5000余人(截至2023年9月),高级职称1200余人,先后培养中国工程院院士3人,拥有雷达信号处理全国重点实验室等2个国家级创新平台和7个省部级创新平台,科技创新持续助力企业发展。

二十三所构建了贯穿需求管理、设计研发、试验测试、装备保障的全流程数字化研发体系,拥有5个现代化生产车间,建设了T/R模块微组装生产线等17条自动化智能生产线。拥有航天广通(北京市“专精特新”)、航天微电(国家级“专精特新”)、航天南湖(科创版上市公司,股票代码:688552、国家级“专精特新”)、航天新气象(国家级“专精特新”)、平湖实验室公司、航天博目六家下属公司,西安、成都两个异地研发中心,21个分支机构遍布全国,形成“北京总部+两个异地研发机构+三大产业集群”的全国区域布局。
\subsubsection{中国航天二院203所}
中国航天科工集团二院203所(即北京无线电计量测试研究所),创建于一九五七年十二月五日。航天203是以计量测试技术为总体,以溯源性研究为主线,以发展计量校准技术、频率控制技术、自动测试技术、综合保障技术为支柱的国防技术基础研究所。

203是国防科技工业计量科研项目管理办公室;是计量校准技术国家级重点实验室;是航天电磁兼容检测中心;是航天科工集团二院武器装备综合保障工程技术研究中心;是中国军用电子元器件质量认证委员会计量机构;是国家、国防ISO9001标准认证单位;是国家、国防、军队检测/校准/计量认可实验室;是航天产品用晶体元器件定点供应单位,拥有贯标生产线;是中国宇航学会计量与测试专业委员会主任委员单位;主办《宇航计测技术》(全国中文核心期刊);主办《国防科技工业计量简报》;是信号与信息处理专业硕士学位授权单位。
\subsection{主要业务方向}

二院研制的交会对接雷达和回收雷达多次保障载人航天、探月工程等国家重大航天工程任务;成功研制北斗导航系统核心设备原子钟;圆满完成北京奥运会、上海世博会、北京冬奥会等安保科技系统建设与运行任务,树立了“航天安保”品牌;自主研发基于龙芯、飞腾等技术路线五代51款产品,信创市场占有率行业领先;成为国内规模最大的气象雷达供应商;承担国家新一代智慧海防试点任务,苏州、赣州、新余等多地智慧城市建设项目,确立了在国内智慧城市建设领域的主导地位。突破呼吸机关键技术,为抗击疫情贡献航天力量。

军贸市场呈现“产品系列化、渠道多元化、营销体系化”良好态势,形成了以“飞龙”“飞獴”“飞豹”“快狼”“野牛”五大品牌为代表的多个系列军贸产品发展型谱。以安保科技、气象雷达、应急通讯及光机电微波类特色产品为代表的民品出口持续增长。

\subsection{和自动化学科的结合点}

中国航天科工二院在自动化学科的结合点上具有广泛的应用和发展潜力。自动化技术在该院的多个业务方向中发挥了重要作用,具体体现在以下几个方面:

\begin{itemize}
    \item \textbf{导弹系统的自动化控制}:二院的导弹系统中,自动化控制技术被广泛应用于导弹发射、导航、跟踪以及打击精度的提升。通过引入先进的自动化控制系统,能够提高导弹的作战效能,优化作战流程,增强整体系统的智能化水平。
    \item \textbf{雷达系统的自动化}:交会对接雷达和回收雷达等关键技术中,自动化系统的应用使得雷达的目标探测、跟踪、数据处理等过程更加高效和精确。自动化技术在雷达信号处理、目标识别和信息融合方面提供了重要支持,大大提升了雷达系统的可靠性和实时性。
    \item \textbf{智能制造与生产自动化}:在高端装备制造和现代服务业中,自动化技术的引入提升了生产线的智能化水平,实现了生产过程的精确控制和高效管理。二院在装备制造中使用先进的自动化设备和生产线,不仅提升了生产效率,也提高了产品质量和一致性。
    \item \textbf{智能城市建设}:在智慧城市建设项目中,自动化技术被应用于城市管理、交通控制、环境监测等领域。自动化系统帮助实现了城市管理的智能化和信息化,提升了城市运行的效率和质量。
    \item \textbf{航天器自动化系统}:在载人航天和探月工程中,自动化技术对航天器的导航、控制、数据采集等功能至关重要。自动化系统的应用使得航天器在复杂环境下能够自适应变化,提高了任务的成功率和航天器的操作安全性。
\end{itemize}

\subsection{座谈交流问题}

\begin{enumerate}
    \item 在这个日新月异的时代,您认为,我们当代大学生应该如何准备,才能迎接时代的挑战,为国家做出自己的贡献?
    \item 中国航天科工二院在高科技领域取得了显著成就。对于有志于从事航天科技的学生,您认为哪些技能和知识是最为重要的?如何有效地提升这些技能?
    \item 自动化技术在各个领域中的应用越来越广泛。您认为未来自动化技术的发展趋势是什么?二院如何利用自动化技术进一步推动其业务和技术的发展?
    \item 在国际化经营和市场拓展方面,二院已经取得了一定成绩。对于未来的国际市场,二院计划采取哪些策略以进一步拓展国际业务?这对有意向从事国际化工作的年轻人来说,有什么建议?
    \item 中国航天科工二院在技术创新方面有哪些成功的经验?对于希望从事科研工作的学生,您有什么建议以激发他们的创新能力?
    \item 在实现高质量发展的过程中,二院是如何应对技术挑战和市场需求变化的?这些经验对年轻一代科技工作者有何启示?
    \item 您如何看待人工智能和大数据在航天科技领域的应用前景?二院在这方面的研究和应用有哪些具体案例?
\end{enumerate}

\end{document}
