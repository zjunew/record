\documentclass[a4paper,12pt]{article} 
\usepackage{ctex,abstract,graphicx}
\usepackage{subfigure,float,fancyhdr}
\usepackage[a4paper, total={6in, 8in}]{geometry}

\pagestyle{fancy}
\fancyhf{}
\fancyfoot[C]{\thepage}
\rhead{王德茂}
\lhead{Numerical Calculation Homework}

\begin{document}

\title {数值计算大作业}
\author{王德茂  \\学号:3220105563  \\
指导老师:徐祖华}

\date{\today}
\maketitle

\pagenumbering{Roman}
\renewcommand{\abstractname}{\textbf{\zihao{4}摘\quad 要}}
\begin{onecolabstract}
  在本次作业中,笔者选取了测量太湖面积的题目。在使用常见的多项式拟合方法后,尝试了使用蒙特卡洛算法,取得了良好的效果。

  
  \addcontentsline{toc}{section}{摘要}\tolerance=500 % 将摘要添加到目录中

  \noindent{\textbf{关键词:}数值,算法,拟合}
\end{onecolabstract}
\renewcommand{\abstractname}{\textbf{Abstract}}
\begin{onecolabstract}
    In this assignment, the author selected the topic of measuring the area of the 
    Taihu Lake Lake. After using common polynomial fitting methods, 
    author attempted to use the Monte Carlo algorithm and achieved good results.

  \noindent{\textbf{Keywords:}Numerical value, Algorithm, Fitting}
\end{onecolabstract}
\newpage
\begin{center}
  \tableofcontents
\end{center}


\newpage
\pagenumbering{arabic}

\section{绪论}
\subsection{研究背景}



\subsection{研究目的}

\subsection{研究意义}







\section{数据预处理}
\subsection{数据采集}

\subsection{数据预处理}


\section{指标的构建}
\subsection{变量探索性分析}

\subsection{变量筛选}


\noindent{\textbf{\zihao{4}【小结】}}


\noindent{\textbf{\zihao{4}【参考文献】}}


[1] 基于R语言的蒙特卡洛法在不确定度评定中的应用	黄欢,黄宇 [化学分析计量] 2022-02-20

[2] Numerical Computing with MATLAB [美] Cleve B.Moler

[3] 25 CLASSICAL METAHEURISTICS --FROM DESING TO MATLAB IMOLEMENTATION 崔建双


\addcontentsline{toc}{section}{参考文献}\tolerance=500 
\addcontentsline{toc}{section}{小结}\tolerance=500
\end{document}
